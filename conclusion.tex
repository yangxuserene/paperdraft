\section{Conclusions}
\label{sec:conclusion}
In this work, we have studied the communication behavior of three parallel applications, 
namely AMG, CrystalRouter and MultiGrid. 
Each application has a distinctive communication pattern, 
which can be representative for a whole range of jobs commonly seen in HPC environment. 
We have used the CODES simulation toolkit  
to simulate the running of these applications on a torus network. 

We have analyzed the intra- and inter-job interference by simulating three applications
running both independently and concurrently.
Based on our comprehensive experiments, we made six observations. 

1) Compact allocation may not be necessary for every application.   
2) The applications dominated by nearest neighbor communications exhibit 
    relatively stable performance under different allocation shapes as long as
    the allocation exhibit some degree of locality.
    
3) The applications dominated with many-to-many communication exhibit 
    better performance with more compact allocations (e.g, 3D balanced).
    
4) A good rank-to-node mapping strategy can greatly improve 
    an application performance when a specific allocation is given.
    
5) An optimal size for allocation units should be determined 
    according to an application's dominant communication pattern. 
    In general, a unit size should be large enough to accommodate  
    neighboring communication in the application. 
    
6) Inter-job interference is inevitable in non-contiguous allocation. 
    However, choosing the proper allocation unit size with communication 
    pattern awareness can help alleviate the resulting negative effects. 

We believe that the findings in this work can provide valuable guidance 
for HPC batch scheduler and resource manager to make flexible job allocations. 
Rather than using pre-defined partitions or non-contiguous placement policy, 
future HPC systems should assign resources to each job based on job communication patterns. 


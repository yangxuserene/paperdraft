\section{Conclusions}
\label{sec:conclusion}
In this work, we have studied the communication behavior of three parallel applications, namely AMG, CrystalRouter and MultiGrid. Each application has distinctive communication pattern, that can be representative for a group of jobs in HPC workload. We have used a sophisticate simulation tool named CODES from Argonne National Laboratory to simulate the running of these three parallel applications on torus network. The torus network provided by CODES has good fidelity and scalability. We have analyzed the performance of each application's communication in terms of data transfer time by simulating them running on torus networks with different bandwidth and dimensionality configurations. We have found that higher dimensionality of torus network would improve the performance of application with ``Many-to-Many" communication patterns, while application with intensive local communication like ``Nearest Neighbor" won't benefit much from higher dimensionality.

We have analyzed the intra- and inter-job interference by simulating three applications running on 3D torus network. Based on our comprehensive experiments, we got five observations. 1) The compact allocation may not guarantee the best performance for every application. 2)The applications dominated with Nearest Neighbor communications exhibit relatively stable performance under different allocation shapes, while the applications dominated with Many-to-Many communication exhibit better performance with compact allocation (e.g. 3D balanced). 3) A good rank-to-node mapping strategy can greatly improve application's performance when specific allocation is given. 4) An optimal size for allocation units should be determined according to an application's dominant communication pattern. In general, a unit size should be large enough to preserve neighboring communication of the application. 5) Inter-job interference is inevitable in non-contiguous allocation. However, choosing the proper allocation unit size with job's communication pattern awareness can alleviate such negative effect.


We believe that our finding in this work can provide valuable guidance for HPC resource management to make flexible job allocations. Rather than using pre-defined partitions and non-contiguous allocation, future HPC systems should assign each job with preferable resources based on job's communication pattern. 


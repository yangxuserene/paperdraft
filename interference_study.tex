\section{Interference Analysis}
\label{sec:interference}

\textbf{\emph{Intra-job interference}} refers to the network 
contention between the ranks within each application. 
\textbf{\emph{Inter-job interference}} is introduced by concurrently 
running jobs share network resources. 
Communication variability due to such interference can 
cause application performance degradation.

In this section we will study both kinds of interference 
in the context of a torus network. 
The current generation of IBM Blue Gene/Q (BG/Q) supercomputers, 
such as Mira at Argonne Nation Laboratory and 
Sequoia at Lawrence Livermore National Laboratory, 
have their compute nodes connected by a 5D torus network~\cite{bgq}. 
The K computer from Japan uses the ``Tofu'' interconnected system, 
which has a 6D mesh/torus topology~\cite{tofu}. 
Titan, a Cray XK7 supercomputer located at the Oak Ridge Leadership Computing Facility (OLCF), 
has nodes connected in a 3D torus within the compute partition~\cite{titan}. 
Since the applications communication patterns do not change with the scale,
we perform our experiments with comparatively (to leadership class) modest scale,
simulating a 3D torus network with 2048 nodes ($16\times 16 \times 8$) to simplify analysis.
%we use AMG with 216 ranks, CrystalRouter with 100 ranks, and MultiGrid 125 ranks. 
%And in order to accommodate three applications running concurrently with different allocations, 
%we simulate a 2K-node 3D torus ($16^{2} \times 8$) network for the experiments.


\subsection{Intra-Job Interference Analysis}
\label{sec: introjob}

We design two sets of experiments to study the 
intra-job interference of each application. 
In the first, 
we assign each application with allocations in \emph{three different shapes}. 
In the second, 
we study the intra-job interference by using different 
mapping strategies for each application with a given allocation shape. 



\subsubsection{Allocation Shape Study}


\begin{figure}[t]
    \centering
        \includegraphics[height=2in]{figs/allocshape/allocation}
        \caption{Contiguous allocation in three different shapes. 
    ``Blue" is a 2D mesh, ``Green" is a 3D-unbalanced cube and ``Red" is a 3D-balanced cube. }
        \label{fig:cont_sub1}
\end{figure}

In the allocation shapes experiment, 
we select three possible shapes commonly seen on the 3D torus network. 
They are 3D balanced-cube, 3D unbalanced-cube and 2D mesh, 
as shown in Figure \ref{fig:cont_sub1}.
3D balanced-cube, shown in red in Figure \ref{fig:cont_sub1}, 
can guarantee the minimum \emph{average pair-wise distance} within the allocation. 
Some research studies~\cite{leung,abhinav-sc13} indicate that 
compact allocation can guarantee jobs with better performance. 
They use a variety of metrics to evaluate the compactness of the allocation, 
such as \emph{average pair-wise distance}, \emph{diameter} and \emph{contiguity}. 
In this work, we select 3D balanced-cube as the most compact allocation on a 3D torus network.

3D unbalanced-cube, shown in green in Figure~\ref{fig:cont_sub1}, 
is a rectangular prism, which is a possible allocation shape 
on the systems with asymmetric networks. 
For example, Cray XE6/XK7 systems are deployed 3D tori with Gemini routers. 
The network connections in the \emph{y}-direction have 
only half the bandwidth of the cables used in the \emph{x} and \emph{z} directions. 
In order to take advantage of the faster links in the \emph{x} and \emph{z} directions, 
job allocation favors the X-Z plane~\cite{RF}. Our torus network in this study is symmetric.

2D mesh, shown in blue in Figure \ref{fig:cont_sub1}, 
can be cut out from a single layer of the 3D torus. 
2D mesh is a common allocation shape on the torus network 
for both contiguous and non-contiguous placement policies. 
For example, Cray's Application Level Placement Scheduler (ALPS) indexes 
all compute nodes in the torus into a list and 
allocates by simply going through that list~\cite{carl-cug}. 
When the list is obtained by sorting the nodes based on 
their spacial coordinates in the torus, 
the resulting allocations form 2D meshes. 
The IBM Blue/Gene Q supercomputer Mira at Argonne Leadership Computing Facility 
also allows its allocation partition to be configured as a mesh~\cite{zhou-ipdps}. 


\begin{figure*}[t]
    \centering
    \begin{subfigure}[t]{0.32\textwidth}
        \centering
        \includegraphics[height=1.8in]{figs/intra-job/shapestudy/amg_box}
        \caption{AMG}
        \label{fig:shapstudy-amg}
    \end{subfigure}%
    \begin{subfigure}[t]{0.32\textwidth}
        \centering
        \includegraphics[height=1.8in]{figs/intra-job/shapestudy/cr_box}
        \caption{CrystalRouter}
        \label{fig:shapstudy-cr}
    \end{subfigure}
    \begin{subfigure}[t]{0.32\textwidth}
        \centering
        \includegraphics[height=1.8in]{figs/intra-job/shapestudy/mg_box}
        \caption{MultiGrid}
        \label{fig:shapstudy-mg}
    \end{subfigure}%
   \caption{
   Data transfer time of AMG, CrystalRouter and MultiGrid on 3D-balanced, 
   3D-unbalanced and 2D allocation.
   }
   \label{fig:shapestudy}
\end{figure*}


Figure~\ref{fig:shapstudy-amg} shows 
AMG with 3D nearest neighbor communication pattern takes slightly less time 
when being assigned with 3D-unbalanced allocation than 3D-balanced. 
And MultiGrid performs best (shortest data transfer time) 
when running on 3D-balanced allocation, as shown in Figure~\ref{fig:shapstudy-mg}. 
Since MultiGrid's communication pattern is many-to-many dominant, 
3D-balanced is the most compact allocation with shortest pair-wise distance between nodes, 
which can reduce the aggregated hops for transferring message among ranks in MultiGrid. 
Since CrystalRouter exibits both local and global rank-to-rank data transfers, 
the 3D-balance is also the best allocation, but its advantage over 3D-unbalanced is not 
that obvious as to MultiGrid, as shown in Figure~\ref{fig:shapstudy-cr}.



A number of studies have designed complex placement algorithms 
to provide applications with the most compact allocation~\cite{leung,LO}. 
However, as shown in our experiments, providing cubic allocation without 
considering application's communication pattern may not guarantee 
the best performance for every application. 
Compact allocation should be provided to applications with intensive global data transfer, 
such as those exhibiting a many-to-many pattern. 






\subsubsection{Mapping Strategy Study}
The rank-to-node mapping of parallel applications on HPC systems 
could greatly impact applications' performance. 
However, finding the optimal mapping solution for a given 
application is out the scope of this work. 
Our experiments aims to show how mapping strategies impact the 
intra-job interference of application with specific communication pattern.

We provide AMG, CrystalRouter and MultiGrid with 3D-balance allocation and 
use three mapping strategies to do the rank-to-node mapping. 
The ``Linear" mapping, as we used in the previous experiments, 
maps each rank according to the dimensional ordering of compute nodes. 
The ``Cube" mapping assigns ranks into consecutive $2^{3}$ cubes. 
The ``Random" mapping assigns ranks randomly within the allocation. 

AMG behavior remains roughly the same when it been mapped by ``Linear" and ``Cube", 
shown in Figure~\ref{fig:diffmap-amg}. 
Although ``Linear" and ``Cube" mapping strategies cause some routing overlap for AMG's communication, 
both mapping strategies still preserve the locality of AMG's 3D nearest neighbor communication pattern. 
The ``Random" mapping disrupts AMG's communication pattern, 
and results in intra-job interference among the ranks. 
The performance degradation caused by using ``Random" mapping is up to 90\%.

The ``Cube" mapping improves CrystalRouter's performance over the ``Linear" 
by up to 10\% on average, shown in Figure~\ref{fig:diffmap-cr}. 
This is due to the global data transfer in CrystalRouter takes fewer hops with ``Cube" mapping. 
The ``Random" mapping for CrystalRouter results in poor locality across all ranks on average. 
of CrystalRouter and make their communication less efficient. 
The ``Cube" mapping benefits MultiGrid's many-to-many communication. 
However, due to the small amount of data transfered among ranks in MultiGrid, 
``Cube" mapping fails to exhibit a drastic advantage. 
``Random" mapping causes little degradation for the same reason, 
shown in Figure~\ref{fig:diffmap-mg}.


\begin{figure*}[t]
    \centering
    \begin{subfigure}[t]{0.32\textwidth}
        \centering
        \includegraphics[height=1.8in]{figs/diffmapping/amg}
        \caption{AMG}
        \label{fig:diffmap-amg}
    \end{subfigure}%
    \hspace{1em}%
    \begin{subfigure}[t]{0.32\textwidth}
        \centering
        \includegraphics[height=1.8in]{figs/diffmapping/cr}
        \caption{CrystalRouter}
        \label{fig:diffmap-cr}
    \end{subfigure}%
    \begin{subfigure}[t]{0.32\textwidth}
        \centering
        \includegraphics[height=1.8in]{figs/diffmapping/mg}
        \caption{MultiGrid}
        \label{fig:diffmap-mg}
    \end{subfigure}%
   \caption{
   Data transfer time of AMG, CrystalRouter and MultiGrid 
   on 3D-balanced allocation using  different mapping strategies.
   }
   \label{fig:diffmap}
\end{figure*}




\subsection{Inter-Job Interference Analysis}
\label{sec: interjob interference study}

Inter-job interference has been identified as one of the major culprits 
for application's performance variability~\cite{abhinav-sc13,skinner}. 
Inter-job interference is a more prominent issue for systems adopting 
non-contiguous placement policies than systems with contiguous policy. 
Application communication times have been demonstrated to 
vary from 36\% shorter to 69\% longer due to job interference 
when they are running concurrently with non-contiguous allocations~\cite{abhinav-sc13}.

We allocate each application with non-contiguous policy and 
run them concurrently on the same network. 
The compute nodes belonging to different jobs are interleaved. 
In order to study the impact of different allocation unit sizes on 
applications' inter-job interference, 
we conduct experiments with unit size of 16, 8 and 2, 
shown respectively in Figure~\ref{fig:noncont_sub1}, 
~\ref{fig:noncont_sub2} and~\ref{fig:noncont_sub3}. 
Figure~\ref{fig:interjobstudy} shows the results of each application 
data transfer time with different allocation unit size. 

\begin{figure*}[t]
    \centering
%    \begin{subfigure}[t]{0.22\textwidth}
%        \centering
%        \includegraphics[height=1.2in]{figs/allocshape/allocation}
%        \caption{ }
%        \label{fig:cont_sub1}
%    \end{subfigure}%
%    \hspace{1em}%
    \begin{subfigure}[t]{0.32\textwidth}
        \centering
        \includegraphics[height=1.8in]{figs/allocshape/unitsize/unit16}
        \caption{ }
        \label{fig:noncont_sub1}
    \end{subfigure}%
    \hspace{1em}%
    \begin{subfigure}[t]{0.32\textwidth}
        \centering
        \includegraphics[height=1.8in]{figs/allocshape/unitsize/unit8}
        \caption{ }
        \label{fig:noncont_sub2}
    \end{subfigure}%
    \hspace{1em}%
    \begin{subfigure}[t]{0.32\textwidth}
        \centering
        \includegraphics[height=1.8in]{figs/allocshape/unitsize/unit2}
        \caption{ }
        \label{fig:noncont_sub3}
    \end{subfigure}%
    \caption{
%    (a) Contiguous allocation in three different shapes. 
%    ``Blue" is a 2D mesh, ``Green" is a 3D-unbalanced cube and ``Red" is a 3D-balanced cube. 
    Non-contiguous allocation. Each job is represented by a specific color. 
    The nodes assigned to different jobs are interleaved, 
    the size of allocation unit are 16, 8 and 2.
    }
    \label{fig:alloc-shapes}
\end{figure*}


The data transfer time of AMG in Figure~\ref{fig:interjob-amg} keeps stable 
between allocation unit size of 16 and 8 because of the 
nearest neighbor pattern of AMG. 
When the unit size is reduced to 2, AMG suffers prolong data transfer time by about 10\% on average. 
CrystalRouter is more sensitive to allocation unit size. 
Figure~\ref{fig:interjob-cr} shows that unit size of 16 and 8 can guarantee 
the same average data transfer time, 
while some ranks spend more time with allocation unit size 8 than 16. 
When the unit size is reduced to 2, the communication becomes less efficient 
and takes about 15\% more time on average for transferring data. 

The data transfer time of MultiGrid with different allocation unit sizes 
doesn't show obvious variability in Figure~\ref{fig:interjob-mg}. 
This is because even big allocation unit size like 16 will still fail to preserve MultiGrid's many-to-many pattern. 
The data transfer time is almost doubled when MultiGrid running concurrently with allocation unit size of 16, 
as shown in Figure~\ref{fig:interjob-mg}. 
Further unit size decreases result in roughly similar average communication times though.

Choosing the proper unit size in non-contiguous placement policy should 
consider application's communication patterns. 
Inter-job interference is inevitable in non-contiguous based systems, 
but unit sizes big enough to preserve the neighborhood communication of the application 
will alleviate such interference and improve job performance. 



\begin{figure*}[t]
    \centering
    \begin{subfigure}[t]{0.32\textwidth}
        \centering
        \includegraphics[height=1.8in]{figs/inter-job/amg}
        \caption{AMG}
        \label{fig:interjob-amg}
    \end{subfigure}%
    \hspace{1em}%
    \begin{subfigure}[t]{0.32\textwidth}
        \centering
        \includegraphics[height=1.8in]{figs/inter-job/cr}
        \caption{CrystalRouter}
        \label{fig:interjob-cr}
    \end{subfigure}%
    \begin{subfigure}[t]{0.32\textwidth}
        \centering
        \includegraphics[height=1.8in]{figs/inter-job/mg}
        \caption{MultiGrid}
        \label{fig:interjob-mg}
    \end{subfigure}%
   \caption{
   Inter-job interference study. 
   ``cont" indicates three applications running side by side concurrently on the same network with contiguous allocation. 
   To study the impact of non-contiguous allocation on inter-job interference, 
   applications running concurrently with interleaved allocation of different unit sizes,
    which are 16-node, 8-node, 2-node. 
    }
   \label{fig:interjobstudy}
\end{figure*}

\subsection{Result Summary}
\label{sec:summary}

Based on our simulation study, we can make following observations:

\begin{itemize}

    \item Compact allocation may not be necessary for every application. 
    
    \item The applications dominated by nearest neighbor communications exhibit 
    relatively stable performance under different allocation shapes as long as
    the allocation exhibit some degree of locality.
    
     
    \item The applications dominated with many-to-many communication exhibit 
    better performance with more compact allocations (e.g, 3D balanced).
    
    \item A good rank-to-node mapping strategy can greatly improve 
    an application performance when a specific allocation is given.
    
    \item An optimal size for allocation units should be determined 
    according to an application's dominant communication pattern. 
    In general, a unit size should be large enough to accommodate  
    neighboring communication in the application. 
    
    \item Inter-job interference is inevitable in non-contiguous allocation. 
    However, choosing the proper allocation unit size with communication 
    pattern awareness can help alleviate the resulting negative effects. 
    
\end{itemize}






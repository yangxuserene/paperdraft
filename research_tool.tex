\section{Research Vehicle}
\label{sec:codes}

It is difficult to accurately and flexibly experiment with concurrently running jobs in an HPC context. 
One reason is that the allocation strategy used on production machine is part of the system software, 
which can not be changed by users. 
Even system administrators may not be able to make such changes. 
Another reason is that it is unrealistic to reserve the system exclusively 
to run the same job with desired allocation without interference and 
then compare the results with those in the presence of interference. 
Therefore, we resort to simulation for this work.


A simulation toolkit named CODES enables the exploration of simulating 
different HPC networks at flit-level with high fidelity~\cite{Jason-2011, mubarak-sc2012}. 
CODES is built on top of the Rensselaer Optimistic Simulation System (ROSS) parallel discrete-event simulator~\cite{ross}, 
which is capable of processing billions of events per second on leadership-class supercomputers. 
CODES support both torus and dragonfly network with high fidelity flit-level simulation.
CODES additionally has the capability of replaying MPI application traces, 
gathered via the SST DUMPI profiler~\cite{sst}.
%CODES has this network workload component that is capable of conducting trace-driven simulations. 
%It can take MPI application traces generated by SST DUMPI  to drive CODES network models. 
%In this work, we focus on an in-depth analysis of intra- and inter-job 
%communication interferences with different job allocations on torus-connected HPC systems. 
%Torus networks have been extensively used in the current generation of supercomputers 
%because of their linear scaling on per-node cost and competitive communication performance.
%


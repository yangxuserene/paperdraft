\section{Related Work}
\label{sec:related_work}

There are many tools available for system monitoring and application profiling. 
Tools like TAU (Tuning and Analysis)~\cite{tau} and mpiP~\cite{mpip} can capture 
application's runtime information and keep records in event traces. 
However, recognizing the patterns about application's communication from those traces requires substantial effort. 
There are a number of studies on the recognition and characterization of parallel application communication patterns. 
Oak Ridge National Laboratory has its ongoing project about developing a tool set named Oxbow, 
which can characterize the computation and communication behavior of scientific applications and benchmarks~\cite{oxbow}. 
In a recent work~\cite{roth}, 
the authors demonstrate a new approach to automatically characterizing parallel applications communication behaviors. 

%\textbf{P2: communication characteristics}\\

Many research efforts have been conducted to characterize scientific applications. 
For instance, a DOE project aims at the identification of the computational characteristics of 
the DOE MiniApps developed at various exascale co-design centers~\cite{designforwardwebpage}. 
In this project, the communication patterns of several DOE full applications and associated 
mini-applications are studied to provide a more complete snapshot of the DOE workload. 
A joint project named CORAL from Oak Ridge, Argonne and Livermore provides a series of 
benchmarks to represent DOE workloads and technical requirements~\cite{coral}. 
The CORAL project includes scalable science benchmarks, throughput benchmarks, 
data centric benchmarks, skeleton benchmarks and Micro benchmarks.  

%\textbf{P3: Interference between concurrently running jobs}\\
The interference among concurrently running jobs on HPC systems 
have been identified as major culprit for job's performance variability. 
Bhatele et al. found that concurrently running applications interfere with each other, 
and cause their communication time varied from 
36\% faster to 69\% slower on different HPC systems~\cite{abhinav-sc13}. 
Skinner et al. found that  there is a 2-3 times of slowdown in MPI$\textunderscore$Allreduce 
due to network contention from other concurrently running jobs~\cite{skinner}. 
%Rosenthal et al. found increasing network bandwidth provide limited benefits to a number of applications. 
%The applications that send mostly small messages or large messages asynchronously 
%are not bandwidth bounded, hence, benefit only slightly or not at all from increased bandwidth~\cite{rosenthal}.

%\textbf{P4: Job allocation(little bit)}\\
Several research studies focus on optimizing job allocation on HPC systems 
to alleviate the interference between concurrently running jobs. 
Hoefler et al. proposed to use performance modeling techniques to analyze factors that 
impact the performance of parallel scientific applications~\cite{hoefler-modeling}. 
However, as the scale of HPC systems continue to grow, 
the interference of concurrently running jobs is getting worse, 
which is hard to be quantified by performance profiling tools alone. 
Bogdan et al. provide a set of guidelines of how to configure a Dragonfly network 
for workload with nearest neighbor communication pattern~\cite{Bogdan-hpdc14}. 
Dong et al. have developed simple benchmarks that conforms to four different communication patterns, 
namely ping-pong, nearest neighbor, broadcast and all$\textunderscore$reduce, 
to demonstrate the effectiveness of this highly parallel 5D torus network~\cite{Dong-SC11}.

Our work is different from all these works in the following ways. 
First, we focus on the dominant communication patterns rather than any specific application. 
We believe this can provide a guideline for other research work. 
Secondly, we explore the intra- and inter-job interference between concurrently running jobs, 
while similar work such as~\cite{abhinav-sc13} only focuses 
on the single application's performance degradation due to network contention. 
Finally, we analyze the impact of different placement strategies to job's communication behaviors. 
We further identify the optimal placement strategy for 
each application with a specific dominant communication pattern. 
Based on our study, we claim that further batch scheduler should take 
job's communication pattern into consideration for its placement decision making. 


\section{Related Work}
\label{sec:related_work}

There are many tools available for system monitoring and application profiling. 
Tools such as TAU (Tuning and Analysis)~\cite{tau} and mpiP~\cite{mpip} can capture 
application runtime information, keeping record in event traces. 
However, recognizing communication patterns from those traces require substantial effort. 
There are a number of studies on the recognition and characterization of parallel application communication patterns. 
Oak Ridge National Laboratory has an ongoing project about developing a tool set named Oxbow, 
which can characterize the computation and communication behavior of scientific applications and benchmarks~\cite{oxbow}. 
In a recent work~\cite{roth}, 
the authors demonstrate a new approach to automatically characterizing parallel applications communication behaviors. 

%\textbf{P2: communication characteristics}\\

Many research efforts have been conducted to characterize scientific applications. 
For instance, the DOE Design Forward Project aims to identify the 
computation and communication characteristics of a collection of relevant 
MiniApps developed at a number of exascale co-design centers~\cite{designforwardwebpage}. 
In this project, the communication patterns of several DOE full applications and associated 
mini-applications are studied to provide a more complete snapshot of DOE workloads. 
A joint project named CORAL from Oak Ridge, Argonne and Livermore provides a series of 
benchmarks to represent DOE workloads and technical requirements~\cite{coral}. 
The CORAL project includes scalable science benchmarks, throughput benchmarks, 
data centric benchmarks, skeleton benchmarks and micro benchmarks.  

%\textbf{P3: Interference between concurrently running jobs}\\
The interference among concurrently running jobs on HPC systems 
have been identified as major a culprit for job's performance variability. 
Bhatele et al. found that concurrently running applications interfere with each other, 
and cause their communication time varied from 
36\% shorter to 69\% longer on different HPC systems~\cite{abhinav-sc13}. 
Skinner et al. found that  there is a 2-3 times of slowdown in MPI$\textunderscore$Allreduce 
due to network contention from other concurrently running jobs~\cite{skinner}. 
%Rosenthal et al. found increasing network bandwidth provide limited benefits to a number of applications. 
%The applications that send mostly small messages or large messages asynchronously 
%are not bandwidth bounded, hence, benefit only slightly or not at all from increased bandwidth~\cite{rosenthal}.

%\textbf{P4: Job allocation(little bit)}\\
Several research studies focus on optimizing job allocation on HPC systems 
to alleviate the interference between concurrently running jobs. 
Hoefler et al. proposed to use performance modeling techniques to analyze factors that 
impact the performance of parallel scientific applications~\cite{hoefler-modeling}. 
However, as the scale of HPC systems continue to grow, 
the interference of concurrently running jobs is getting worse, 
which is hard to quantify with performance profiling tools alone. 
Bogdan et al. provide a set of guidelines on how to configure a Dragonfly network 
for workload with nearest neighbor communication pattern~\cite{Bogdan-hpdc14}. 
Dong et al. have developed simple benchmarks that conforms to four different communication patterns, 
namely ping-pong, nearest neighbor, broadcast and all$\textunderscore$reduce, 
to demonstrate the effectiveness of 5D torus networks~\cite{Dong-SC11}.

We differentiate our work from these works in the following ways. 
First, we focus on the dominant communication patterns rather than any specific application. 
We believe this can provide a guideline for other research work. 
Secondly, we explore both intra- and inter-job interference between concurrently running jobs, 
while similar work such as~\cite{abhinav-sc13} focuses 
on one single application's performance degradation due to network contention. 
Finally, we analyze the impact of different placement strategies to job's communication behaviors, 
identifying preferred placement strategies for 
each application with a specific dominant communication pattern. 
Based on our study, we claim that future batch schedulers should take 
job communication patterns into consideration for placement decision making. 


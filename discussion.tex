
\section{Discussion}
\label{sec:discussion}

The results shown in the previous section provide insights 
for the design of smart and flexible job placement policy. 
By scrutinizing job's communication behavior, 
we can identify job's dominant communication pattern 
and pinpoint the ``neighborhood communication". 
With such knowledge about jobs' communication patterns, 
we can analyze the possible interference between jobs and 
take precautions to alleviate the negative effect when making placement decisions. 

When an application with dominant communication that is intensive many-to-many, 
the scheduler should provide it with compact node allocation and exclusive network provision. 
The compact allocation can guarantee the shortest pair-wise distance between all the ranks, 
and make the data transfer between ranks take less hops. 
On the other hand, 
the exclusive network provision will prevent other adjacent applications from sharing network resources, 
thus eliminate the performance degradation due to interference.

Not every application's preferable resource are compact allocation 
and exclusive network provision. 
As we demonstrate in our study, 
applications whose dominant communication pattern contains intensive 
``neighborhood communication" like nearest Neighbor do not benefit 
from compact allocation and exclusive network provision. 
Applications, such as AMG, can run with non-contiguous node 
allocation without performance degradation, 
as long as the allocation unit can accommodate their ``neighborhood communication".

The size of allocation unit shouldn't be fixed. 
Instead, the scheduler should choose a proper unit size 
based on the scale of ``neighborhood communication" of each job. 
The best unit size should be neither too big nor too small, 
just perfectly fit for the scale of that ``neighborhood". 
Big unit size can not be fully utilized and thus cause fragmentation. 
Small unit size will not be able to accommodate the ``neighborhood communication", 
making the intra-job communication less efficient. 


The advantage of making job placement with consideration about job's communication pattern is obvious. 
First, compared with contiguous allocation policy, 
the new scheduler with job's communication pattern awareness can be more flexible, 
without requiring the system to provide a big contiguous partition to accommodate the whole application, 
it only needs to provide a small set of compact nodes that is 
sufficient for all the "local community" in the application. 
On the other hand, 
the new scheduler is able to preserve the locality of application's communication pattern. 
Indeed, the small compact node set (allocation unit) provided for the application's 
``local communities" can preserve the communication locality in the maximum extent. 
The design of such a new scheduler with job communication pattern awareness 
is part of our future work. 





\section{Discussion}
\label{sec:discussion}

The results shown in the previous sections provide insights 
for the design of a smart and flexible job placement policy. 
By scrutinizing the communication behavior of jobs, 
we can identify their dominant communication patterns 
and pinpoint locality needs. 
With such knowledge about communication patterns, 
we can analyze the possible interference between jobs and 
take precautions to alleviate interference when making placement decisions. 

When an application with intensive many-to-many communication is submitted to the system, 
the scheduler should provide it with compact node allocation and exclusive network resources. 
Compact allocation can guarantee the shortest pair-wise distance between all the ranks. 
Additionally, the exclusive network provision will prevent other jobs from sharing network resources, 
thus eliminating the performance degradation due to interference.

Not every application requires compact allocation and exclusive network provisioning. 
As we demonstrate in our study, 
applications whose dominant communication pattern contains intensive 
``neighborhood communication" like nearest neighbor may not benefit 
from compact allocation and exclusive network provision. 
Applications, such as AMG, can run with non-contiguous node 
allocation without significant performance degradation, 
as long as the allocation unit can accommodate their rank-rank locality.

The size of allocation unit shouldn't be fixed. 
Instead, the scheduler should choose a proper unit size 
based on the granularity of communication locality of each job.   
Large unit sizes can not be fully utilized and thus cause fragmentation. 
Small unit sizes will not be able to accommodate the communication locality, 
resulting in less efficient intra-job communication.


The advantage of performing job placement with consideration 
of job communication patterns is straightforward,
given our experimental analyses. 
Compared with contiguous placement policy, 
schedulers with communication pattern awareness can be more flexible, 
relaxing the need to provide contiguous partitions to accommodate the whole application,
avoiding fragmentation issues inherent in contiguous placement.
Indeed, smaller compact node sets (allocation units) provided for application's
``local communication" can help preserve the communication performance to a sufficient extent. 
The design of such a new scheduler with job communication pattern awareness 
is part of our future work. 



